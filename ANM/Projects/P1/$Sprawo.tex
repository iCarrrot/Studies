\documentclass{article}
\usepackage[utf8]{inputenc}
\usepackage{polski}
\usepackage[polish]{babel}
\usepackage{bbm}
\usepackage{graphicx}    
\usepackage{caption}
\usepackage{subcaption}
\usepackage{epstopdf}
\usepackage{amsmath}
\usepackage{amsthm}
\usepackage{hyperref}
\usepackage{url}
\usepackage{comment}
\newtheorem{defi}{Definicja}
\newtheorem{twr}{Twierdzenie}


\author{Michał Martusewicz 282023}
\date{Wrocław, \today}
\title{\textbf{Rozważania na temat \textit{Metody Steffensena}}  \\ Sprawozdanie do zadania P.1.17}

\begin{document}
\maketitle
\section{Wstęp}

Wybrane przeze mnie zadanie polega na zaprogramowaniu metody Steffensena dla wybranych funkcji.
Metoda ta służy do znajdowania pierwiastków równania nieliniowego i przy pewnych założeniach jest zbieżna kwadratowo.
\\ \indent
 W \S 2 przedstawię ogólnie tą metodę, założenia zbieżności kwadratowej i dowód.
 \\ \indent
 W \S 3 przedstawię wybrane funkcje, ich miejsca zerowe wg wybranych źródeł (min. biblioteki \textit{polynomial} i \textit{WolframAlpha}), ich miejsca zerowe wg zaprogramowanej przeze mnie funkci i ilość znaczących cyfr dokładnych liczonych ze wzoru $-\log_{10}{|x-(\widetilde x)|}$, co pozwoli doświadczalnie sprawdzić efektywność tej metody.
W \S 4 przedstawię przykłady zachowania się tej metody iteracyjnej dla różnych początkowych $x_0$
 
Wszystkie testy numeryczne przeprowadzone są przy użyciu programu \textit{jupyter notebook} w języku \textit{Julia}, symulując tryb precyzji 256-bitowej.

\section{Założenia zbieżności}
\textbf{Metodę Steffensena} można przedstawić w następujący sposób:
$$
x_{k+1}:= x_k - f(x_k)/g(x_k),  \qquad k=0,1,...,
$$

gdzie


$$
g(x):=[f(x+f(x))-f(x)]/f(x)
$$

czyli


$$
x_{k+1}:=x_k-\frac{f^2(x_k)}{f(x_k+f(x_k))-f(x_k)} \qquad k=0,1,...,
$$
Warto również zauważyć, że $g(x)$ jest przybliżeniem $f'(x)$, więc zachowuje się podobnie do metody newtona, ale nie trzeba znać pocodnej funkcji. z drugiej strony metoda Steffensena jest (w ogólnym przypadku) wolniej zbieżna od metody Newtona \cite{bib1}, choć tak jak ona jest zbieżna kwadratowo \cite{bib5} . 

\section{Porównanie wyników}

\section{Rozbieganie metody}
\documentclass{article}
\usepackage[utf8]{inputenc}
\usepackage{polski}
\usepackage[polish]{babel}
\usepackage{bbm}
\usepackage{graphicx}    
\usepackage{caption}
\usepackage{subcaption}
\usepackage{epstopdf}
\usepackage{amsmath}
\usepackage{amsthm}
\usepackage{hyperref}
\usepackage{url}
\usepackage{comment}
\newtheorem{defi}{Definicja}
\newtheorem{twr}{Twierdzenie}


\author{Michał Martusewicz 282023}
\date{Wrocław, \today}
\title{\textbf{Rozważania na temat \textit{Metody Steffensena}}  \\ Sprawozdanie do zadania P.1.17}

\begin{document}
\maketitle
\section{Wstęp}

Wybrane przeze mnie zadanie polega na zaprogramowaniu metody Steffensena dla wybranych funkcji.
Metoda ta służy do znajdowania pierwiastków równania nieliniowego i przy pewnych założeniach jest zbieżna kwadratowo.
\\ \indent
 W \S 2 przedstawię ogólnie tą metodę, założenia zbieżności kwadratowej i dowód.
 \\ \indent
 W \S 3 przedstawię wybrane funkcje, ich miejsca zerowe wg wybranych źródeł (min. biblioteki \textit{polynomial} i \textit{WolframAlpha}), ich miejsca zerowe wg zaprogramowanej przeze mnie funkci i ilość znaczących cyfr dokładnych liczonych ze wzoru $-\log_{10}{|x-(\widetilde x)|}$, co pozwoli doświadczalnie sprawdzić efektywność tej metody.
W \S 4 przedstawię przykłady zachowania się tej metody iteracyjnej dla różnych początkowych $x_0$
 
Wszystkie testy numeryczne przeprowadzone są przy użyciu programu \textit{jupyter notebook} w języku \textit{Julia}, symulując tryb precyzji 256-bitowej.

\section{Założenia zbieżności}
\textbf{Metodę Steffensena} można przedstawić w następujący sposób:
$$
x_{k+1}:= x_k - f(x_k)/g(x_k),  \qquad k=0,1,...,
$$

gdzie


$$
g(x):=[f(x+f(x))-f(x)]/f(x)
$$

czyli


$$
x_{k+1}:=x_k-\frac{f^2(x_k)}{f(x_k+f(x_k))-f(x_k)} \qquad k=0,1,...,
$$
Warto również zauważyć, że $g(x)$ jest przybliżeniem $f'(x)$, więc zachowuje się podobnie do metody newtona, ale nie trzeba znać pocodnej funkcji. z drugiej strony metoda Steffensena jest (w ogólnym przypadku) wolniej zbieżna od metody Newtona \cite{bib1}, choć tak jak ona jest zbieżna kwadratowo \cite{bib5} . 

\section{Porównanie wyników}
Tu będą wykresy i tabelki
\section{Rozbieganie metody}
Tu będą wykresy i tabelki

\section{Wnioski}




\begin{thebibliography}{9}
	\itemsep2pt
	\bibitem{bib1} \url{http://home.agh.edu.pl/~dziembaj/Old/skrypt%20end/podstrony/steffensen.html}
	(ostatni dostęp do strony \today)
	
	\bibitem{bib2} \url{https://en.wikipedia.org/wiki/Steffensen's_method}
	(ostatni dostęp do strony \today)
    \bibitem{bib3} David Kincaid, Ward Cheney. Analiza numeryczna 
    \bibitem{bib3} \url{http://bit.ly/2fgKejw}
	(ostatni dostęp do strony \today)
   \bibitem{bib5} A. Bjorck G. Dahlquist - Metody numeryczne (strona 225)
    
		
\end{thebibliography}

\end{document}

\section{Wnioski}




\begin{thebibliography}{9}
	\itemsep2pt
	\bibitem{bib1} \url{http://home.agh.edu.pl/~dziembaj/Old/skrypt%20end/podstrony/steffensen.html}
	(ostatni dostęp do strony \today)
	
	\bibitem{bib2} \url{https://en.wikipedia.org/wiki/Steffensen's_method}
	(ostatni dostęp do strony \today)
    \bibitem{bib3} David Kincaid, Ward Cheney. Analiza numeryczna 
    \bibitem{bib3} \url{http://bit.ly/2fgKejw}
	(ostatni dostęp do strony \today)
   \bibitem{bib5} A. Bjorck G. Dahlquist - Metody numeryczne (strona 225)
    
		
\end{thebibliography}

\end{document}